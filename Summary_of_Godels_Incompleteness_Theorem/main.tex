\documentclass{article}
\usepackage[utf8]{inputenc}

\title{Summary of Godel's Incompleteness Theorem}
\author{Aditi Bhattacharya }
\date{20 July 2021}

\begin{document}

\maketitle
\section{ Qualitative Summary}
Gödel’s two incompleteness theorems are among the most important results in modern logic, and have deep implications for various issues. Mathematicians and computer scientists all agree that Gödel’s Incompleteness theorem is one of the most important and impactful results in modern logic and mathematics. Kurt Godel changed the world of logic, mathematics and philosophy with his paper — an attempt to mathematise the workings of the human mind by describing them as formal system. They concern the limits of provability in formal axiomatic theories. The first incompleteness theorem states that in any consistent formal system F  within which a certain amount of arithmetic can be carried out, there are statements of the language of F which can neither be proved nor disproved in F. According to the second incompleteness theorem, such a formal system cannot prove that the system itself is consistent (assuming it is indeed consistent). These results have had a great impact on the philosophy of mathematics and logic. There have been attempts to apply the results also in other areas of philosophy such as the philosophy of mind, but these attempted applications are more controversial.

His incompleteness theorems meant there can be no mathematical theory of everything, no unification of what’s provable and what’s true. What mathematicians can prove depends on their starting assumptions, not on any fundamental ground truth from which all answers spring.
\section{Implication on AI}
One of the goals of AI research is to achieve “strong artificial intelligence”, meaning human-level general AI. Currently, we build AI as algorithms in Turing machines, which are consistent axiomatic systems and therefore subject to the theorem.

Roger Penrose and J.R. Lucas argue that human consciousness transcends Turing machines because human minds, through introspection, can recognize their own inconsistencies, which under Gödel’s theorem is impossible for Turing machines. They argue that this makes it impossible for Turing machines to reproduce traits of human minds, such as mathematical insight.

\end{document}
