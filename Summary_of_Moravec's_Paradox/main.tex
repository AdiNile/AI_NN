\documentclass{article}
\usepackage[utf8]{inputenc}
\PassOptionsToPackage{hyphens}{url}\usepackage{hyperref}
\setlength{\parindent}{4em}
\setlength{\parskip}{1em}

\setlength{\emergencystretch}{10pt}

\title{Summary of Moravec's Paradox}
\author{Aditi Bhattacharya}
\date{26 July 2021}

\begin{document}

\maketitle
\begin{sloppypar}
The Moravec's paradox is an idea articulated by Hans Moravec among many others, which states that it is easier to design an artificial system which can perform high-level intelligent tasks, than to make the system perform tasks which generally don't involve cognitive skills when performed by humans.
 \par

A thing to be noted here is the fact that humans, also tend to become pretty bad at these tasks once when they consciously try to perfect the task-in-hand. The intuition of something 'being a part of common sense' seems similar in the entire human society. However, if one sits to justify or deduce how a piece of information becomes a part of common-sense, s/he might end up questioning the very basics on which the entire universe works. Take the famous example of the summation of one and one, which gives a result of two. Sounds quite clear, however, there's a very famous 360 pages prove of this 'common-sense' question.

A paper published recently claimed to have built a large dataset which lets an algorithm to work with intuitive psychology. Intuitive psychology, the ability to reason about
hidden mental variables that drive observable actions, comes naturally to people: even pre-verbal infants can tell agents from objects, expecting agents to act efficiently to achieve goals given constraints. They propose AGENT, a benchmark for core psychology reasoning, which consists of a large-scale dataset of cognitively inspired tasks designed to probe machine agents’ understanding of key concepts of intuitive psychology in four scenarios – Goal Preferences, Action Efficiency, Unobserved Constraints, and Cost-Reward Trade-offs. The fact that it took us so long to build such a sophisticated dataset, very well verifies the Moravec's paradox. 
\newline
\newline
References:
\newline
\url {https://www.storyofmathematics.com/20th\_russell.html\#:\~:text=Some\%20idea\%20of\%20the\%20scope,logic\%20since\%20Aristotle's\%20\%E2\%80\%9COrganon\%E2\%80\%9D
}
\newline
\url{Shu, Tianmin, et al. "AGENT: A Benchmark for Core Psychological Reasoning." arXiv preprint arXiv:2102.12321 (2021).
}
\end{sloppypar}
\end{document}
