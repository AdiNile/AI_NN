\documentclass{article}
\usepackage[utf8]{inputenc}
\PassOptionsToPackage{hyphens}{url}\usepackage{hyperref}
\setlength{\parindent}{4em}
\setlength{\parskip}{1em}

\title{Busting Myth revolving around Vaimānika Shāstra}
\author{Aditi Bhattacharya }
\date{5 August 2021}

\begin{document}

\maketitle

\section*{Introduction}
The Vaimānika Śāstra is an early 20th-century text in Sanskrit. It makes the claim that the vimānas mentioned in ancient Sanskrit epics were advanced aerodynamic flying vehicles.
\newline
The existence of the text was revealed in 1952 by G. R. Josyer who asserted that it was written by Pandit Subbaraya Shastry (1866–1940), who dictated it during the years 1918–1923. A Hindi translation was published in 1959, while the Sanskrit text with an English translation was published in 1973. It contains 3000 shlokas in 8 chapters which Shastry claimed was psychically delivered to him by the ancient Hindu sage Bharadvaja. The text has gained favour among proponents of ancient astronauts.
\newline
Vaimānika Shāstra and vimanas are also mentioned in works about pseudoscience such as Regal's Pseudoscience: A Critical Encyclopedia. According to Regal, vimanas are one of the common attempts to fit elements of ancient cultures into contemporary narratives.

\section*{Possible Use of Artificial Intelligence to Bust the Myth}
Although the approach might seem abstract and skeptical, but may lead to conclusive results:
\begin{itemize}
\item Design a Natural Language Processor which can learn Sanskrit and reproduce it to high precision.  Millions of original Sanskrit texts (apart from the ones involving the mention of Vaimānika Shāstra) can be used to train the algorithm. It has to unsupervised in its initial stages. 
\item Use the precision and accuracy of the algorithm when put in competition against Sanskrit scholars to implement reinforcement learning. 
\item Once the algorithm is standardised to the levels of a scholar, provide the algorithm with the data of Vaimānika Shāstra.
\item Design a simulation using the data provided by the algorithm to check whether such an object can soar across the sky.

\end{itemize}
\section*{Loopholes in the Protocol}
The most threatening loophole is the contamination of the dataset with misinformation and dubious Sanskrit translations.
There is no known way of checking whether the Sanskrit scholars indeed have the pristine knowledge of the context in hand. This can lead our algorithm to become biased and hence the results wouldn't lead us to meaningful inference.

\section*{References:}
\url{https://en.wikipedia.org/wiki/Vaim\%C4\%81nika_Sh\%C4\%81stra}


\end{document}
